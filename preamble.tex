% height = 5.9in width = 4.37in
% make this variable
%% \usepackage[paperheight = 4.37in, paperwidth = 5.9in, margin=0.2in]{geometry}
\usepackage[a4paper, margin=1in]{geometry}

\usepackage[export]{adjustbox}
\usepackage{graphicx}
\usepackage{xcolor}
\usepackage{circuitikz}
\usepackage{subfiles}
\usepackage{amsmath, amssymb}
\usepackage{enumitem}
\usepackage{nicematrix}
\usepackage{minted}
\usepackage{caption}
\usepackage{lmodern}
\usepackage{bookmark}
\usepackage{tabularx}
\usepackage{multirow}
\usepackage{multicol}
\usepackage{booktabs}
\usepackage{titlesec}
\usepackage{xspace}
\usepackage{varwidth}
\usepackage{titletoc}
\usepackage{epigraph}
\usepackage{etoolbox}
\usepackage{fontawesome5}
\usepackage[style = ieee]{biblatex} %Imports biblatex package
\usepackage{xfp}
\usepackage{xifthen}
\usepackage{tcolorbox}
\usepackage{xltabular}
\usepackage[T1]{fontenc}
\usepackage{setspace}
\usepackage[numbib]{tocbibind}

%% for bold textsc to work
\rmfamily % To load Latin Modern Roman and enable the following NFSS declarations.
% Declare that Latin Modern Roman (lmr) should take
% its bold (b) and bold extended (bx) weight, and small capital (sc) shape,
% from the corresponding Computer Modern Roman (cmr) font, for the T1 font encoding.
\DeclareFontShape{T1}{lmr}{b}{sc}{<->ssub*cmr/bx/sc}{}
\DeclareFontShape{T1}{lmr}{bx}{sc}{<->ssub*cmr/bx/sc}{}

\addbibresource{references.bib} %Import the bibliography file

\usepackage[bottom]{footmisc}

\usepackage{gensymb}
\usepackage{siunitx}
\usepackage{pgfplots}

\pgfplotsset{
    compat=newest,
    colormap={black}{rgb255=(0,0,0) rgb255=(0,0,0)}
}

\usetikzlibrary{intersections}
\usetikzlibrary{positioning}
\usetikzlibrary{calc}
\usetikzlibrary{ext.topaths.arcthrough}
\usetikzlibrary{decorations.markings}
\usetikzlibrary{patterns}
\usetikzlibrary{patterns.meta}

\setlength{\arraycolsep}{0pt}
\renewcommand\arraystretch{1.5}

%% custom packages

\usepackage{colorscheme}
\usepackage{subtikzpicture}
\usepackage{generalcommands}

% alerts

\newcommand\alertCaution[1] {
    \begin{tcolorbox}[
            coltitle = colorAlertCaution,
            colbacktitle = colorAlertBgCaution,
            colback = colorAlertBgCaution,
            colframe = colorAlertBgCaution,
            title=Caution,
            fonttitle=\bfseries,
            detach title,
        ]
        \begin{minipage}[t]{0.18\textwidth}
            \begin{flushleft}
                \tcbtitle
            \end{flushleft}
        \end{minipage}
        \begin{minipage}[t]{0.8\textwidth}
            #1
        \end{minipage}
    \end{tcolorbox}
}

\newcommand\alertWarning[1] {
    \begin{tcolorbox}[
            coltitle = colorAlertWarning,
            colbacktitle = colorAlertBgWarning,
            colback = colorAlertBgWarning,
            colframe = colorAlertBgWarning,
            title=Warning,
            fonttitle=\bfseries,
            detach title,
        ]
        \begin{minipage}[t]{0.18\textwidth}
            \begin{flushleft}
                \tcbtitle
            \end{flushleft}
        \end{minipage}
        \begin{minipage}[t]{0.8\textwidth}
            #1
        \end{minipage}
    \end{tcolorbox}
}

\newcommand\alertImportant[1] {
    \begin{tcolorbox}[
            coltitle = colorAlertImportant,
            colbacktitle = colorAlertBgImportant,
            colback = colorAlertBgImportant,
            colframe = colorAlertBgImportant,
            title=Important,
            fonttitle=\bfseries,
            detach title,
        ]
        \begin{minipage}[t]{0.18\textwidth}
            \begin{flushleft}
                \tcbtitle
            \end{flushleft}
        \end{minipage}
        \begin{minipage}[t]{0.8\textwidth}
            #1
        \end{minipage}
    \end{tcolorbox}
}

\newcommand\alertTip[1] {
    \begin{tcolorbox}[
            coltitle = colorAlertTip,
            colbacktitle = colorAlertBgTip,
            colback = colorAlertBgTip,
            colframe = colorAlertBgTip,
            title=Tip,
            fonttitle=\bfseries,
            detach title,
        ]
        \begin{minipage}[t]{0.18\textwidth}
            \begin{flushleft}
                \tcbtitle
            \end{flushleft}
        \end{minipage}
        \begin{minipage}[t]{0.8\textwidth}
            #1
        \end{minipage}
    \end{tcolorbox}
}

\newcommand\alertNote[1] {
    \begin{tcolorbox}[
            coltitle = colorAlertNote,
            colbacktitle = colorAlertBgNote,
            colback = colorAlertBgNote,
            colframe = colorAlertBgNote,
            title=Note,
            fonttitle=\bfseries,
            detach title,
        ]
        \begin{minipage}[t]{0.18\textwidth}
            \begin{flushleft}
                \tcbtitle
            \end{flushleft}
        \end{minipage}
        \begin{minipage}[t]{0.8\textwidth}
            #1
        \end{minipage}
    \end{tcolorbox}
}

%% for paragraphs

\setlength{\parskip}{0.5\baselineskip}

%% for fontawesome

%for scaling of fontawesome
\DeclareFontFamily{U}{fontawesome1}{}
\DeclareFontShape{U}{fontawesome1}{m}{n}{<->FontAwesome--fontawesomeone}{}
\DeclareFontFamily{U}{fontawesome2}{}
\DeclareFontShape{U}{fontawesome2}{m}{n}{<->FontAwesome--fontawesometwo}{}
\DeclareFontFamily{U}{fontawesome3}{}
\DeclareFontShape{U}{fontawesome3}{m}{n}{<->FontAwesome--fontawesomethree}{}
\DeclareFontFamily{U}{fontawesome5}{}
\DeclareFontShape{U}{fontawesome5}{m}{n}{<->FontAwesome--fontawesomefive}{}
\DeclareRobustCommand{\FAone}{\usefont{U}{fontawesome1}{m}{n}}
\DeclareRobustCommand{\FAtwo}{\usefont{U}{fontawesome2}{m}{n}}
\DeclareRobustCommand{\FAthree}{\usefont{U}{fontawesome3}{m}{n}}
\DeclareRobustCommand{\FAfive}{\usefont{U}{fontawesome5}{m}{n}}

\titleformat{\chapter} [display]
{\bfseries\normalfont\huge\filright\sffamily\vspace{-2cm}}
{\Large\textsc{chapter \num[minimum-integer-digits = 2]{\thechapter}} \vspace{1em}}
{1pc}
{\titlerule\vspace{0.5em}\scshape}
[\vspace{0.5em}{\titlerule[1pt]}]


\titleformat{\section}
{\vspace{-0.35em}\normalfont\Large\filright\sffamily\scshape}
{\thesection}
{8pt}
{\scshape}
{}

\titleformat{\subsection}
{\vspace{-0.35em}\normalfont\large\filright\sffamily\scshape}
% {}
% {0pt}
{\thesubsection}
{8pt}
{\scshape}
{}

\titleformat{\subsubsection}
{\vspace{-0.35em}\normalfont\large\filright\sffamily\scshape}
% {}
% {0pt}
{\thesubsubsection}
{8pt}
{\scshape}
{}

%\titleformat{\section} [display]
%{\bfseries\normalfont\large\filright\sffamily}
%{}
%{2pt}
%{\scshape}
%{}

\setlength\epigraphwidth{9cm}
\setlength\epigraphrule{0pt}

\renewcommand{\epigraphflush}{center}

\def\esp{\textsc{ESP32}\xspace}
\def\espcam{\textsc{ESP32 CAM}\xspace}

\def\IGXMaxWidth{\textwidth}
\def\IGXMaxHeight{\textheight}
\def\IGXDefaultOptionalArgs{keepaspectratio}

% for figures

\makeatletter
\def\fps@figure{htbp}
\makeatother
