\documentclass[../main]{subfiles}

\input{chapter_header.tex}

\begin{document}

\chapter{SINC: Overview}

Successful seed germination requires precise environmental conditions including
temperature, humidity, light, and air quality. Variations in these parameters
can result in delayed germination, uneven growth, or even total failure.
Traditional manual monitoring is not only labor-intensive but also prone to
human error, leading to inconsistent results. To address this challenge, the
Seed Incubation Environmental Control App has been developed, providing a
digital platform for real-time monitoring and control of incubation
environments.

\section{Project Idea and Objective}

The project’s objective is to integrate hardware and software into a seamless
system that allows both visualization and control of incubation parameters. The
ESP32 microcontroller, paired with multiple sensors, measures environmental
data and transmits it wirelessly to the Flutter-based mobile application. Users
can then monitor the current conditions and modify thresholds that
automatically trigger actuators such as heaters, fans, and humidifiers. The
goal is to reduce human intervention, improve accuracy, and ensure optimal seed
growth through automation.

\section{Importance of Environmental Control for Seed Incubation}

Temperature regulates metabolic rates in seeds, humidity ensures adequate
moisture absorption, light drives photosynthesis, and air quality affects
overall growth. Small deviations in these parameters can significantly impact
germination success. Manual supervision is often inconsistent and cannot
guarantee continuous control. Automating the monitoring and control process
ensures that optimal conditions are maintained at all times, reducing risk and
improving reliability for both research and commercial applications.

\section{Scope of the Project}

The project demonstrates a scalable solution for environmental monitoring,
which can be extended to greenhouses, hydroponics, and vertical farming. Beyond
mobile control, the app architecture allows integration with cloud services for
historical data storage, analytics, and AI-driven predictive models for
intelligent control. This system exemplifies how combining IoT and mobile
technology enhances agricultural management efficiency.

\end{document}
