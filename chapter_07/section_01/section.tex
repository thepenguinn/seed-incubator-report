\documentclass[../../main]{subfiles}

\input{section_header.tex}

\begin{document}

\section{Air Moisture System} \label{sec:}

The air moisture plays an important role in \emph{nurturing} the seedling after the \emph{germination}
completes. So it is also necessary to properly maintain the air moisture inorder to strengthen the
\emph{seedling}. The \emph{Air Moisture} system takes care of regulating the \emph{relative humidity} of the Incubator.
Following sections will go through the \emph{sensing} and \emph{controlling} parts of the \emph{air moisture}
system.

\subsection{Air Moisture Sensing System}

The \emph{air moisture monitoring system} also make use of \emph{DHT22} paired sensors
as described in section \ref{sec:tempMonitoringsystem}. Physically these two
systems\footnote{thermal and air moisture systems.} make use of same hardware,
but conceptually they can be seen as two different systems.

Air moisture specification\footnote{for general and temperature related specification
see section \ref{sec:tempMonitoringsystem}}:

\begin{itemize}
    \item \textbf{Humidity range:} $0-100\%$RH with $2\%$ tolerance.
    \item \textbf{Humidity resolution:} $0.1\%$RH.
\end{itemize}

\alertNote{
    Refer figures \ref{fig:dht22PinImage}, and \ref{fig:absThermalSensorSystem} from chapter
    \ref{chp:thermalExhaustSystem} for \emph{DHT22} pinout and interfacing respectively.
}

\subsection{Air Moisture Control System}

The \emph{air moisture control system} make use of \emph{piezoelectric} humidifiers.
As they are placed in water and powered, they can vibrate in the \emph{ultrasonic} ranges,
and sprays water as particles into the air.

\alertNote{
    These humidifiers can only increase the humidity of the atmosphere, in order to decrease
    the relative humidity, they need to work with the \emph{exhaust system} mentioned in the
    chapter \ref{chp:thermalExhaustSystem}.
}

\subsubsection{Piezoelectric Humidifiers}

\begin{center}
    {\begin{minipage} [c] {0.55\textwidth}

        These \emph{piezoelectric} humidifiers have two parts, one is the \emph{piezoelectric disc}
        itself and the other is the driving circuit. These \emph{discs} have a \emph{resonant}
        frequency. And we need to drive them in this specific frequency to get the maximum output.
        Figure \ref{fig:piezoHumidifierImage} shows the module.

        Specification:

        \begin{itemize}
            \item \textbf{Operating voltage:} $4.5\si{V}$ to $5\si{V}$.
            \item \textbf{Resonant frequency:} $108\si{kHz} \pm 3\si{kHz}$.
            \item \textbf{Spray quantitiy:} $380 \si{mL/h}$.
            \item \textbf{Work power:} $2\si{W}$.
        \end{itemize}

    \end{minipage}
    \hfill
    \begin{minipage} [c] {0.35\textwidth}
        \centering
        \includegraphics [
            max width = \IGXMaxWidth,
            max height = \IGXMaxHeight,
            \IGXDefaultOptionalArgs,
        ] {pics/humidifier.png}
        \captionof{figure} {
            Piezoelectric humidifier module.
            \label{fig:piezoHumidifierImage}
        }
    \end{minipage}\hfill}
\end{center}

\subsubsection{Interfacing}

The driver module is just a $555$ timer circuit. And if we look closely at figure \ref{fig:piezoHumidifierImage}
we can see a push button. If we press once\footnote{given that the module is powered and in off state.}, the
module will turn on. And if we press twice\footnote{given that the module was already on.}, the module will
turn off. If we \emph{trace} the PCB \emph{tracing}, we can see that the push button is simply connecting the
some of the pins to ground, when it is being pressed.

With that information we can now turn on and turn off the module using one of the pins of the
main board\footnote{of the auxiliary system.}. We simply
keep the pin high and keep the \emph{state} of the humidifier \emph{internally}\footnote{as in some variable.}.
When we want to turn on the module, we can simply give a \emph{low pulse} from this pin. And if we want
to turn off the module, we can simply give two pulses.

Refer figure \ref{fig:absHumidifierSystem} for the interfacing of the module, using auxiliary system.

\begin{figure}
    \centering
    \includegraphics [
        max width = \IGXMaxWidth,
        max height = \IGXMaxHeight,
        \IGXDefaultOptionalArgs,
    ] {tikzpics/endAbsHumidifierSystem.pdf}
    \captionof{figure} {Interfacing of piezoelectric humidifier modules.}
    \label{fig:absHumidifierSystem}
\end{figure}

\alertWarning{
    \texttt{P0}, and \texttt{P1} of the main board shown in figure \ref{fig:absHumidifierSystem} are temporary.
    We haven't decided the exact pins that are going to control these piezoelectric humidifiers.
}

\end{document}
