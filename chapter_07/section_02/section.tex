\documentclass[../../main]{subfiles}

\renewcommand\thesection{\arabic{section}}


\begin{document}

\section{Air Quality System} \label{sec:}

The incubator's insides is an isolated system. Therefore the air inside could get
stagnant and become not favorable for the seedlings if not replenished periodically.
This system aims to monitor the $\mbox{CO}_2$ content of the insides and replenish
the air as it gets below a certain threshold.

\subsection{Air Quality Monitoring}

\emph{SCD40} is a preferable $\mbox{CO}_2$ sensor that can sense $\mbox{CO}_2$ in
$400\si{ppm}$ to $2000\si{ppm}$ range.

\alertWarning{
    \emph{SCD40} is an expensive sensor, it cost about $2700$ INR. There are other
    sensors that costs less but they are not NDIR\footnote{Non-Dispersive Infrared} based ones.
    Those sensors are unreliable and requires further calibration to work properly.
    It is possible to not use a sensor at all. All we have to do is periodically
    replenish the air instead of waiting for the $\mbox{CO}_2$ get below the preferred
    threshold. Considering this, we might trade in the air quality sensory system for a battery
    backup system.
}

\subsection{Air Quality Control}

Air quality system will make use of thermal / exhaust system's \texttt{EMODE} to
effectively replenish the air. See chapter \ref{chp:thermalExhaustSystem} to know
more about core block, exhaust branches and exhaust inlet.

\end{document}
