\documentclass[../main]{subfiles}

\input{chapter_header.tex}

\begin{document}

\chapter{EMCU: Overview} \label{chp:}

As the name suggests, the \textsc{Environment Monitoring and Control Unit} is responsible
for taking care of the environmental
aspects of the incubator. EMCU aims to monitor and control aspects like the
\emph{temperature}, \emph{humidity}, \emph{lighting}, \emph{air quality}, etc.

\begin{figure}
    \centering
    \includegraphics [
        max width = \IGXMaxWidth,
        max height = \IGXMaxHeight,
        \IGXDefaultOptionalArgs,
    ] {tikzpics/endAbsEMCU.pdf}
    \captionof{figure} {An abstracted block diagram of EMCU and the aspects it controls.}
    \label{fig:abstractEMCU}
\end{figure}

As we can see from figure \ref{fig:abstractEMCU}, \textsc{Emcu} needs to
\emph{interface} with various systems that have various \emph{actuators} and \emph{sensors}. As a matter of
fact, the sheer amount of \emph{actuators} and \emph{sensors} required to
achieve this exceeds the total number of GPIOs\footnote{General Purpose Input
Output pins.} \esp has. In order to \emph{synchronize} all of the
\emph{subsystems}, \esp needs an \emph{external subsystem} to control all the
other \emph{subsystems}, an \textsc{Auxiliary System} as we will be calling
this \emph{external system}, in this entire document. To know more about auxiliary
system, jump right to chapter \ref{chp:auxiliarySystem}.

Along with this, EMCU interfaces with GMU for accomplishing various tasks. And SINC
app helps the user to control and monitor most of the aspects of EMCU.

% \begin{figure}
%     \centering
%     \includegraphics [
%         max width = \IGXMaxWidth,
%         max height = \IGXMaxHeight,
%         \IGXDefaultOptionalArgs,
%     ] {tikzpics/endEMCUOverview.pdf}
%     \captionof{figure} {Overview EMCU.}
%     \label{fig:emcuOverview}
% \end{figure}

% From figure \ref{fig:emcuOverview}, we can see that the \esp

%% TODO: this should go to the introduction / overview.
% %Different Parts of the Incubator%
% \subfile{section_01/section.tex}

%At the Heart of EMCU%
\subfile{section_01/section.tex}


\end{document}
