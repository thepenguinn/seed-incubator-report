\documentclass[../../main]{subfiles}

\renewcommand\thesection{\arabic{section}}


\begin{document}

\section{Lighting System} \label{sec:}

Proper \emph{lighting} of the Incubator is essential for the proper growth of
the seedlings. The closed in nature of the Incubator enclosure demands some
form of lighting for the growth monitoring of the saplings.

\subsection{Grow Light}

Sunlight that reaches the Earth has a wide range of frequencies to it. This includes
the light from UV spectrum. Inorder to mimic that, we need a light source that has
a wide spectrum. Full spectrum strip LEDs available in the market are ideal for our
purpose. The low profile nature and $5\si{V}$ driving voltage makes easier to interface.

\subsection{Lighting for GMU}

GMU\footnote{Growth Monitoring Unit} uses \espcam module to capture images of the plants.
Then this image data is processed for further analysis. So it is important to capture
good quality images. Using \emph{grow lights} won't result in good images as these type
of lights will have a violet glow to them. We need extra white lights for this purpose.

\subsection{Interfacing Circuits}

\begin{figure}
    \centering
    \includegraphics [
        max width = \IGXMaxWidth,
        max height = \IGXMaxHeight,
        \IGXDefaultOptionalArgs,
    ] {tikzpics/endAbsLightControlSystem.pdf}
    \captionof{figure} {Interfacing of lights through auxiliary system's main board.}
    \label{fig:lightInterface}
\end{figure}

Figure \ref{fig:lightInterface} shows the interfacing of these five light sources. They are
done through the auxiliary system. One of the main board of the auxiliary system will be controling
these five lights.

\alertWarning{
    \texttt{P0}, \texttt{P1}, \texttt{P2}, \texttt{P3}, and \texttt{P4} of the main board
    are temporary. We haven't decided the exact pins that are going to control these lights.
}

\end{document}
