\subtikzpicturedef{subAbsLightControlSystem} {
    origin%
} {
    \draw (#1-start) coordinate (#1-origin);
    \subAbsCtkDiagramBase {#1-absCtkDiagramBase} {#1-origin} {origin}

    \draw
    (#1-origin) coordinate (#1-main)
    ;

    \subAbsMainFour {#1-main} {#1-main} {center}

    \draw
    (#1-main-l0) ++(-3, 0) coordinate (#1-pa0)
    (#1-main-l1) ++(-3, 0) coordinate (#1-pa1)
    (#1-main-l2) ++(-3, 0) coordinate (#1-pa2)
    (#1-main-l3) ++(-3, 0) coordinate (#1-pa3)

    (#1-main-r0) ++(3, 0) coordinate (#1-pa4)
    ;

    \subAbsPlainActuatorModule {#1-pa0} {#1-pa0} {r0}
    \subAbsPlainActuatorModule {#1-pa1} {#1-pa1} {r0}
    \subAbsPlainActuatorModule {#1-pa2} {#1-pa2} {r0}
    \subAbsPlainActuatorModule {#1-pa3} {#1-pa3} {r0}

    \subAbsPlainActuatorModule {#1-pa4} {#1-pa4} {l0}

    \draw
    (#1-main-r3) ++(1, 0) coordinate (#1-esp)
    ;

    \subAbsESP {#1-esp} {#1-esp} {l5}

    \draw
    (#1-esp-l4 -| #1-pa0-l0) coordinate (#1-b0)
    ;

    %% labelling

    \ovrAbsESPDrawLabelName {#1-esp} {\texttt{ESP}}

    \ovrAbsPlainActuatorModuleDrawLabelName {#1-pa0} {GMU L01}
    \ovrAbsPlainActuatorModuleDrawLabelName {#1-pa1} {GMU L02}
    \ovrAbsPlainActuatorModuleDrawLabelName {#1-pa2} {GMU L03}
    \ovrAbsPlainActuatorModuleDrawLabelName {#1-pa3} {GMU L04}

    \ovrAbsPlainActuatorModuleDrawLabelName {#1-pa4} {GROW L01}

    \ovrAbsMainDrawLabelName {#1-main} {MAIN}

    %% connections

    \draw
    (#1-esp-l3 -| #1-main-r0) coordinate (#1-emc0)
    (#1-esp-l4 -| #1-main-r0) coordinate (#1-emc1)
    (#1-esp-l5 -| #1-main-r0) coordinate (#1-emc2)
    ;

    \foreach \start/\stop in {
        #1-pa0-r0/#1-main-l0,
        #1-pa1-r0/#1-main-l1,
        #1-pa2-r0/#1-main-l2,
        #1-pa3-r0/#1-main-l3,
        %%
        #1-pa4-l0/#1-main-r0,
        %%
        #1-esp-l3/#1-emc0,
        #1-esp-l4/#1-emc1,
        #1-esp-l5/#1-emc2%
    } {
        \draw [
            absCtkInnerConnection,
        ]
        (\start) -- (\stop)
        ;
    }

    \draw [
        absCtkAuxModule
    ]
    (#1-main-l0) ++(\absCtkPinPad, 0) node [right = 4pt] {\texttt{P0}}
    (#1-main-l1) ++(\absCtkPinPad, 0) node [right = 4pt] {\texttt{P1}}
    (#1-main-l2) ++(\absCtkPinPad, 0) node [right = 4pt] {\texttt{P2}}
    (#1-main-l3) ++(\absCtkPinPad, 0) node [right = 4pt] {\texttt{P3}}

    (#1-main-r0) ++(-\absCtkPinPad, 0) node [left = 4pt] {\texttt{P4}}
    ;

    %% annotation

    %\draw
    %(#1-leftVert) ++(-0.25, 0) coordinate (#1-leftVert)
    %;
    %
    %\node at (#1-esp-center -| #1-rightVert) [
    %    absCtkAnnotation,
    %    anchor = west,
    %] {
    %    \absCtkAnnotationWrap{
    %    \item \textbf{ESP:} Esp32.
    %    }
    %};
    %
    %\node at (#1-main-center -| #1-rightVert) [
    %    absCtkAnnotation,
    %    anchor = west,
    %] {
    %    \absCtkAnnotationWrap{
    %    \item \textbf{MUX:} Analog Multiplexer, part of the Auxiliary System.
    %    }
    %};
    %
    %\draw ($(#1-pa1-center)!0.50!(#1-pa2-center)$) coordinate (#1-tmp);
    %\node at (#1-tmp -| #1-leftVert) [
    %    absCtkAnnotation,
    %    anchor = east,
    %] {
    %    \absCtkAnnotationWrap{
    %    \item \textbf{LDR:} Light Dependent Resistors.
    %    }
    %};
    %
    %\node at (#1-b0-center -| #1-leftVert) [
    %    absCtkAnnotation,
    %    anchor = south east,
    %] {
    %    \absCtkAnnotationWrap{
    %    \item \textbf{BUF:} Buffer or Enabling Circuit.
    %    \item \textbf{0/1:} One Bit Register, part of the Auxiliary System.
    %    }
    %};

}

\subtikzpictureactivate{subAbsLightControlSystem}
