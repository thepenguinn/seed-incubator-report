\documentclass[../main]{subfiles}

\input{chapter_header.tex}

\begin{document}

\chapter{Auxiliary System} \label{chp:auxiliarySystem}

Now let's tackle the \emph{pin problem} of \esp. Due to
the sheer amount of pins required to \emph{sense} and \emph{actuate} different
parts of the \emph{incubator}, we need a \emph{system} that intermediates this
task. Thus the \emph{birth} of \emph{Auxiliary system}.

The sensory systems is mediated by two set of \emph{Analog multiplexers} which are
\emph{serially} addressable\footnote{right now only one set is implemented.}. And the
actuator systems are mediated by \emph{Shift register} boards. Following sections will
take a deep dive into the design of each of these systems.

\begin{figure}
    \centering
    \includegraphics [
        max width = \IGXMaxWidth,
        max height = \IGXMaxHeight,
        \IGXDefaultOptionalArgs,
    ] {tikzpics/endEMCUOverview.pdf}
    \captionof{figure} {Overview EMCU's auxiliary system.}
    \label{fig:emcuAuxilarySystem}
\end{figure}

As we can see from figure \ref{fig:emcuAuxilarySystem}, we can see that the sensors
and actuators are mediated by three types of boards, namely \textsc{Multiplexer Boards},
\textsc{Relay Boards}, and \textsc{Main Boards}.

%Buffer / Inverter%
\subfile{section_01/section.tex}

%NAND Gate%
\subfile{section_02/section.tex}

%Multiplexer System%
\subfile{section_03/section.tex}

%Shift Register System%
\subfile{section_04/section.tex}

\end{document}
