\subtikzpicturedef{subClvlInverter} {
    in, out, vcc, gnd,
    origin%
} {
    \draw (#1-start) coordinate (#1-origin);
    \subClvlCtkDiagramBase {#1-clvlCtkDiagramBase} {#1-origin} {#1-origin}

    \draw [
        clvlCtkModule,
    ]
    node (#1-npn) [npn] {}
    (#1-npn.C) to[R, name = #1-rc] ++(0, 1.8)
    (#1-npn.B) to[R, name = #1-rb] ++(-1.3, 0)

    (#1-npn.E) -- ++(0, -0.5) coordinate (#1-gnd)
    (#1-rc.east) coordinate (#1-vcc)
    (#1-npn.C) -- ++(1.5, 0) coordinate (#1-out)
    (#1-rb.east) -- ++(-0.5, 0) coordinate (#1-in)
    ;

}

\subtikzpictureactivate{subClvlInverter}

%% subtikzname npnname rc rb
\newcommand\ovrClvlInverterDrawLabelName[4] {
    \path [clvlCtkModule]
    (#1-npn.C) -- (#1-npn.E)
    node [
        midway,
        right = 4pt,
    ] {#2}
    ;

    \path [clvlCtkModule]
    (#1-rc.west) -- (#1-rc.east)
    node [
        midway,
        right = 12pt,
    ] {#3}
    ;

    \path [clvlCtkModule]
    (#1-rb.west) -- (#1-rb.east)
    node [
        midway,
        below = 12pt,
    ] {#4}
    ;
}

%% subtikzname vcc
\newcommand\ovrClvlInverterDrawPwr[2] {
    \draw [clvlCtkModule]
    (#1-vcc) node [vcc] {#2}
    (#1-gnd) node [ground] {}
    ;
}

%% subtikzname
\newcommand\ovrClvlInverterDrawPole[1] {
    \draw [clvlCtkModule]
    (#1-in) node [ocirc] {}
    (#1-out) node [ocirc] {}
    ;
}
