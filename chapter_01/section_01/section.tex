\documentclass[../../main]{subfiles}

\input{section_header.tex}

\begin{document}

Seed Incubation Plant aims to automate the incubation process of a seed in
hope of minimizing the labour cost and improving the yield. The entire
system can be divided into three main sub-systems.

\begin{figure}
    \centering
    \includegraphics[
        max width = 0.8\textwidth,
        max height = 0.8\textheight,
        \IGXDefaultOptionalArgs,
    ] {tikzpics/endIncOverviewUpdated.pdf}
    \captionof{figure} {An overview of the entire incubator}
    \label{fig:incOverView}
\end{figure}

\begin{itemize}
    \item \textbf{\texttt{EMCU}}: \textsc{Environment Monitoring and Control Unit}.
    \item \textbf{\texttt{GMU}}: \textsc{Growth Monitoring Unit}.
    \item \textbf{\texttt{SINC}}: \textsc{Seed IncDroid} (An Android Application).
\end{itemize}

% \subsection{EMCU}
%
% The Environment Monitoring and Control Unit handles the \emph{incubation} process
% and the \emph{well-being} of the seed. As the name suggests, it \emph{monitors} and
% \emph{collects} the real time data from the incubator enclosure and its environment.
% And \emph{controls} these factors to ensure the \emph{optimal} growth of the seedling.
% EMCU tries to monitor and control\footnote{except in the case of fertilizer}
% five different aspects that governs the well-being of the sapling. They include:

% \begin{itemize}
%     \item Temperature.
%     \item Humidity.
%     \item Lighting.
%     \item Soil Moisture.
%     \item Fertilizer.
% \end{itemize}

% Despite taking care of the seedlings, the EMCU performs another major task,
% generating \textsc{alerts}. These \textsc{alerts} are generated when the
% EMCU encounters some scenarios which requires manual interventions, as:

% \begin{itemize}
%     \item The sapling becomes ready to be harvested.
%     \item The water or fertilizer reservoir has been depleted.
%     \item Some of the controlling parameters fails to be regulated.
%     \item The EMCU encounters some unrecoverable errors.
% \end{itemize}

% \subsection{SINC}
%
% SINC is an Android App that tries to bridge the gap between the user and the
% Incubator. It enables the user to configure the Incubator easily and monitor
% the collected data. The App consists of two main sub-systems:
%
% \begin{itemize}
%     \item The UI of the App that visualizes the collected data.
%     \item The system that enables the communication between the App and the
%         Incubator.
% \end{itemize}

\end{document}
