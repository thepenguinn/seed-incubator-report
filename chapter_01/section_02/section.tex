\documentclass[../../main]{subfiles}

\input{section_header.tex}

\begin{document}

\section{Environment Monitoring and Control Unit} \label{sec:}

\textsc{Emcu} tries to monitor and control various aspects that affects the
seed growth. Inorder to accomplish that, \textsc{Emcu} needs to
\emph{interface} various \emph{actuators} and \emph{sensors}. As a matter of
fact, the sheer amount of \emph{actuators} and \emph{sensors} required to
achieve this exceeds the total number of GPIOs\footnote{General Purpose Input
Output pins.} \esp has. In order to \emph{synchronize} all of the
\emph{subsystems}, \esp needs an \emph{external subsystem} to control all the
other \emph{subsystems}, an \textsc{Auxiliary System} as we will be calling
this \emph{external system}, in this entire document.

Apart from the \textsc{Auxiliary System}, the \textsc{Emcu} consists of $5$
other main \emph{hardware subsystems}.

%% PIC

\begin{figure}
    \centering
    \includegraphics [
        max width = 0.5\textwidth,
        max height = 0.5\textheight,
        \IGXDefaultOptionalArgs,
    ] {pics/endAbsSubsystems.pdf}
    \captionof{figure} {Hardware Subsystems of Incubator.}
    \label{fig:absSubsystems}
\end{figure}



%Thermal System%
%Air Moisture System%
%Lighting System%
%Soil Moisture and Fertilizer System%
%Reservoir System%

%% Communication System

%\subsection{Thermal System}
%
%The \emph{thermal system} as the name suggests, takes care of the temperature inside
%the incubator. There are several

\end{document}
